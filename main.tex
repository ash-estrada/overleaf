% \propto

\documentclass[10pt]{article}
\usepackage[utf8]{inputenc}
\usepackage{amsmath}
\setcounter{secnumdepth}{0}
\setlength{\parindent}{0pt}
\setlength{\parskip}{0pt plus 0.5ex}
\usepackage[hmargin=1in,vmargin=1in]{geometry} 
\begin{document}

\textbf{4.3 Homogeneous Linear Equations With Constant Coefficients} \\

\bigskip

Definition: \\
\smallskip
\begin{equation}
a_{n}\frac{d^n y}{dx^n} + a_{n-1}\frac {d^{n-1}y}{dx^{n-1}} + \cdots + a_{2}\frac{d^2y}{dx^2} + a_1\frac{dy}{dx} + a_{0}y = 0
\label{eqn:1}
\end{equation}

\bigskip 

Recall: \\
\begin{align*}
\frac{dy}{dx} = 2y \quad & \Leftrightarrow \int\frac{dy}{y} = 2\int{dx}\\
& \Leftrightarrow \ln{y}\ln{c}  = 2x-\ln{c} \\ 
& \Leftrightarrow ???? \\ 
& \Leftrightarrow \ln{\frac{y}{c}} = 2x \\
& \Leftrightarrow  \frac{y}{c} = e^{2x}\\ 
& \Leftrightarrow y = ce^{2x} \\ 
\end{align*}

From this example, we can infer solutions of (1) are exponential functions. \\

Now, consider \\

\begin{equation}
ay''+by'+cy = 0
\end{equation}
\bigskip

Try a solution of the form \(y = e^{mx}\). \\

\[y=e^{mx},\quad y'=me^{mx}, \quad y''=m^2e^{mx}\]

\[e^{mx}(am^2+bm+c) = 0\] \\

Since \(e^{m^x} \ne 0\), then \((am^2 + bm + c) = 0\), a quadratic equation in \textit{m}. \\

This quadratic equation, \(am^2 + bm + c = 0\), is what we call the \textit{auxiliary equation}. Sometimes, this is called the \textit{characteristic equation}. \\

Consider the following three cases: \\

Case 1: Discriminant \((b^2-4ac) > 0\) \\

Expect two distinct real roots or solutions. \\

Example: \textit{Discuss in class.} \\

Solve:

\[2\frac{d^2y}{dx^2} + 11\frac{dy}{dx} + 4y = 0\]

%%%%%%%%%%%%%
% in class
%%%%%%%%%%%%%

\[m= \frac{-b\pm\sqrt{b^2-4ac}}{2a} \Rightarrow D = b^2-4ac\]

% fill in with handwritten stuff

Case 2: Discriminant \((b^2-4ac) = 0\) \\

Expect one root with multiplicity of 2. That is, the two roots are the same, (i.e. repeated roots, e.g. \(m=m_1=m_2)\). \\

Example in class. \\

Solve:

\[\frac{d^2y}{dx^2} - 18 \frac{dy}{dx}+ 81y = 0 \]

\bigskip
\bigskip
\bigskip

Case 3: Discriminant \((b^2-4ac) = 0\) \\

Expect two distinct complex (imaginary) roots.
\[m_1 = \alpha + i\beta \qquad m_2 = \alpha - i\beta \]

\[y = c_1e^{(\alpha + i\beta)x} + c_2e^{(\alpha - i\beta)x}\] \\

\(e^{i\theta} = \cos\theta + i\sin\theta\) \\

\(e^{i\beta x} = \cos\beta x + i\sin\beta x\) \\

\(e^{-i\beta x} = \cos\beta x - i\sin\beta x\) \\

\(\cos(-\beta x) = \cos(\beta x) \) \\

\(\sin(-\beta x) = -i\sin(\beta x) \)

Solution Form (Always) \\

\(y = e^{\alpha x}[c_1 \cos \beta x + c_2 \sin \beta x] \) \\

\bigskip

Example: Solve \(2\frac{d^2 y}{dx^2} + 3\frac{dy}{dx} + 13y = 0\) \\

\bigskip

\bigskip 

Examples on: Higher-Order Linear DE with Constant Coefficients (Homogeneous) 

Follow discussions in class. 



\end{document}
